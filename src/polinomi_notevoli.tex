\documentclass[a4paper, 12pt]{scrartcl}

\usepackage[italian]{babel}
\usepackage{amsmath, amssymb, amsfonts}

\usepackage{caption}
\usepackage{booktabs}
\usepackage{tabularx}

\usepackage{hyperref}

\usepackage{fontspec}
\usepackage[usefilenames]{plex-otf}
\renewcommand*\familydefault{\sfdefault}

\usepackage{unicode-math}
\setmathfont[Scale=MatchUppercase]{Asana Math}

\captionsetup{
    font=small,
    labelfont=bf,
    textfont=it,
    margin=10pt,
    format=plain
}

\hypersetup{
    colorlinks=true,
    linkcolor=blue,
    filecolor=magenta,      
    urlcolor=cyan,
    pdftitle={Overleaf Example},
    pdfpagemode=FullScreen,
}

\author{Stefano Pilosio}
\title{Appunti di Algebra}
\date{\today}

\begin{document}

\maketitle
\tableofcontents

\section{Polinomi Notevoli}
\begin{table}[h!]
\centering
\caption{Polinomi notevoli}
\begin{tabular}{ccc}
    \toprule
    \textbf{Nome} & \textbf{Espressione} & \textbf{Scomposizione} \\
    \midrule
    Quadrato di un binomio & $ a^2 + 2ab + b^2$ & $(a + b)^2$ \\
    Quadrato di un trinomio & $a^2 + b^2 + c^2 + 2ab + 2ac + 2bc$ & $(a + b + c)^2$ \\
    Cubo di un binomio & $a^3 + 3a^2b + 3ab^2 + b^3$ & $(a + b)^3$ \\
    Somma per differenza & $a^2 - b^2$ & $(a + b)(a - b)$ \\ 
    Somma di cubi & $a^3 + b^3$ & $(a + b)(a^2 - ab + b^2)$ \\ 
    Differenza di cubi & $a^3 - b^3$ & $(a - b)(a^2 + ab + b^2)$ \\
    \bottomrule
\end{tabular}
\end{table}

\section{Regola di Ruffini}
La regola di Ruffini è un metodo rapido per eseguire la divisione di un polinomio \( P(x) \) per un binomio della forma \( (x - x_0) \), dove \( x_0 \) è una radice del polinomio.

\subsection*{Procedura}
\begin{enumerate}
    \item Scrivi i coefficienti del polinomio \( P(x) \) in ordine decrescente rispetto ai gradi. Se mancano termini, inserisci uno zero come coefficiente.
    \item Scrivi il valore di \( x_0 \) (la radice) a sinistra di una linea verticale.
    \item Riporta il primo coefficiente sotto la linea orizzontale.
    \item Moltiplica il valore sotto la linea per \( x_0 \) e scrivi il risultato sotto il coefficiente successivo.
    \item Somma i valori nella colonna e scrivi il risultato sotto la linea.
    \item Ripeti i passaggi 4 e 5 fino a completare tutti i coefficienti.
\end{enumerate}

\subsection*{Risultato}
Alla fine:
\begin{itemize}
    \item I numeri sotto la linea rappresentano i coefficienti del quoziente.
    \item L'ultimo numero a destra è il resto della divisione.
\end{itemize}

\subsection*{Esempio}
Dividiamo \( P(x) = 2x^3 + 3x^2 - 5x + 6 \) per \( x - 2 \):
\[
\begin{array}{r|rrr|r}
   & 2 & 3 & -5 & 6 \\
2  &   & 4 & 14 & 18 \\
\hline
  & 2 & 7 & 9  & 24 \\
\end{array}
\]
Il quoziente è \( 2x^2 + 7x + 9 \) e il resto è \( 24 \).

\end{document}